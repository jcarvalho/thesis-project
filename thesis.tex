\documentclass{llncs}
\usepackage[utf8]{inputenc}
\usepackage{cite}
%
\begin{document}

\title{LongXact: Making Long-Lived Transactions Easier to Develop}

%% Title
\author{João Carvalho}
\institute{Técnico de Lisboa/Universidade Técnica de Lisboa
\email{joao.pedro.carvalho@ist.utl.pt}}
\maketitle

%% Abstract

\begin{abstract}
Over the past years, Software Transactional Memories have become more
and more popular, growing to be more than a research topic. On top of
that, the concept has been extended to encompass persistence, so
the concept of Persistent Software Transactional Memories was born. In
this thesis, I propose an extension to PSTM's in order to support the
concept of Long-Lived Transactions. Some research on the topic has
been conducted in the field of Database Transactions and Workflow
systems. My thesis is that supporting Long-Lived Transactions should
be done at the infrastructural level on Persistent STM's, so this
paper will describe how those systems can be extended to support
Long-Lived Transactions.
\end{abstract}

%% Intro

\section{Introduction}

For many years, enterprise applications were developed using
two-tiered architectures. In such architectures, there was typically a
mainframe with great computational power.

Long-Lived Transactions were first described in
1981 as ``[..] transactions with lifetimes of a few days or
weeks''\cite{gray1981transaction}. At the time, the author said that the
solution for this 

\section{Long-Lived Transactions}

\subsection{What are Long-Lived Transactions?}

\subsection{Why are they difficult to implement?}

Mention Tx Checks in the middle of regular application code.

Isolation breach, keep a paralel domain copy, or add logic to the
domain.

     - Quebra de isolamento, ou manter uma copia paralela ao dominio
     (ou aumentar o dominio para contemplar o estado intermedio)

\subsection{Applications of LLTs}

\subsection{Database Management Systems}

\subsection{Workflow Systems}

\subsection{Object-Oriented}

\section{Related Work}

\subsection{Database world}

\subsection{Workflow Management Systems}

\subsection{Object-Relational Mappings}

\section{Solution Architecture}

In this section I will describe the planned architecture 

\subsection{Fenix Framework}

Describe that the solution will be build on top of the Fenix Framework
core, plus JVSTM, making it agnostic to the specific persistence
backend in use. 

\section{Evaluation}

\section{Conclusion}

\bibliography{thesis.bib}
\bibliographystyle{plain}
\end{document}
